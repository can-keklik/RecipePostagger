
%% bare_conf.tex
%% V1.4b
%% 2015/08/26
%% by Michael Shell
%% See:
%% http://www.michaelshell.org/
%% for current contact information.
%%
%% This is a skeleton file demonstrating the use of IEEEtran.cls
%% (requires IEEEtran.cls version 1.8b or later) with an IEEE
%% conference paper.
%%
%% Support sites:
%% http://www.michaelshell.org/tex/ieeetran/
%% http://www.ctan.org/pkg/ieeetran
%% and
%% http://www.ieee.org/

%%*************************************************************************
%% Legal Notice:
%% This code is offered as-is without any warranty either expressed or
%% implied; without even the implied warranty of MERCHANTABILITY or
%% FITNESS FOR A PARTICULAR PURPOSE! 
%% User assumes all risk.
%% In no event shall the IEEE or any contributor to this code be liable for
%% any damages or losses, including, but not limited to, incidental,
%% consequential, or any other damages, resulting from the use or misuse
%% of any information contained here.
%%
%% All comments are the opinions of their respective authors and are not
%% necessarily endorsed by the IEEE.
%%
%% This work is distributed under the LaTeX Project Public License (LPPL)
%% ( http://www.latex-project.org/ ) version 1.3, and may be freely used,
%% distributed and modified. A copy of the LPPL, version 1.3, is included
%% in the base LaTeX documentation of all distributions of LaTeX released
%% 2003/12/01 or later.
%% Retain all contribution notices and credits.
%% ** Modified files should be clearly indicated as such, including  **
%% ** renaming them and changing author support contact information. **
%%*************************************************************************


% *** Authors should verify (and, if needed, correct) their LaTeX system  ***
% *** with the testflow diagnostic prior to trusting their LaTeX platform ***
% *** with production work. The IEEE's font choices and paper sizes can   ***
% *** trigger bugs that do not appear when using other class files.       ***                          ***
% The testflow support page is at:
% http://www.michaelshell.org/tex/testflow/



\documentclass[conference]{IEEEtran}
% Some Computer Society conferences also require the compsoc mode option,
% but others use the standard conference format.
%
% If IEEEtran.cls has not been installed into the LaTeX system files,
% manually specify the path to it like:
% \documentclass[conference]{../sty/IEEEtran}



\renewcommand\IEEEkeywordsname{Keywords}

% Some very useful LaTeX packages include:
% (uncomment the ones you want to load)


% *** MISC UTILITY PACKAGES ***
%
%\usepackage{ifpdf}
% Heiko Oberdiek's ifpdf.sty is very useful if you need conditional
% compilation based on whether the output is pdf or dvi.
% usage:
% \ifpdf
%   % pdf code
% \else
%   % dvi code
% \fi
% The latest version of ifpdf.sty can be obtained from:
% http://www.ctan.org/pkg/ifpdf
% Also, note that IEEEtran.cls V1.7 and later provides a builtin
% \ifCLASSINFOpdf conditional that works the same way.
% When switching from latex to pdflatex and vice-versa, the compiler may
% have to be run twice to clear warning/error messages.


 % \usepackage[turkish]{babel}
   \usepackage[latin5]{inputenc}
  % \usepackage[T1]{fontenc}




% *** CITATION PACKAGES ***
%
%\usepackage{cite}
% cite.sty was written by Donald Arseneau
% V1.6 and later of IEEEtran pre-defines the format of the cite.sty package
% \cite{} output to follow that of the IEEE. Loading the cite package will
% result in citation numbers being automatically sorted and properly
% "compressed/ranged". e.g., [1], [9], [2], [7], [5], [6] without using
% cite.sty will become [1], [2], [5]--[7], [9] using cite.sty. cite.sty's
% \cite will automatically add leading space, if needed. Use cite.sty's
% noadjust option (cite.sty V3.8 and later) if you want to turn this off
% such as if a citation ever needs to be enclosed in parenthesis.
% cite.sty is already installed on most LaTeX systems. Be sure and use
% version 5.0 (2009-03-20) and later if using hyperref.sty.
% The latest version can be obtained at:
% http://www.ctan.org/pkg/cite
% The documentation is contained in the cite.sty file itself.






% *** GRAPHICS RELATED PACKAGES ***
%
\ifCLASSINFOpdf
  % \usepackage[pdftex]{graphicx}
  % declare the path(s) where your graphic files are
  % \graphicspath{{../pdf/}{../jpeg/}}
  % and their extensions so you won't have to specify these with
  % every instance of \includegraphics
  % \DeclareGraphicsExtensions{.pdf,.jpeg,.png}
\else
  % or other class option (dvipsone, dvipdf, if not using dvips). graphicx
  % will default to the driver specified in the system graphics.cfg if no
  % driver is specified.
  % \usepackage[dvips]{graphicx}
  % declare the path(s) where your graphic files are
  % \graphicspath{{../eps/}}
  % and their extensions so you won't have to specify these with
  % every instance of \includegraphics
  % \DeclareGraphicsExtensions{.eps}
\fi
% graphicx was written by David Carlisle and Sebastian Rahtz. It is
% required if you want graphics, photos, etc. graphicx.sty is already
% installed on most LaTeX systems. The latest version and documentation
% can be obtained at: 
% http://www.ctan.org/pkg/graphicx
% Another good source of documentation is "Using Imported Graphics in
% LaTeX2e" by Keith Reckdahl which can be found at:
% http://www.ctan.org/pkg/epslatex
%
% latex, and pdflatex in dvi mode, support graphics in encapsulated
% postscript (.eps) format. pdflatex in pdf mode supports graphics
% in .pdf, .jpeg, .png and .mps (metapost) formats. Users should ensure
% that all non-photo figures use a vector format (.eps, .pdf, .mps) and
% not a bitmapped formats (.jpeg, .png). The IEEE frowns on bitmapped formats
% which can result in "jaggedy"/blurry rendering of lines and letters as
% well as large increases in file sizes.
%
% You can find documentation about the pdfTeX application at:
% http://www.tug.org/applications/pdftex





% *** MATH PACKAGES ***
%
%\usepackage{amsmath}
% A popular package from the American Mathematical Society that provides
% many useful and powerful commands for dealing with mathematics.
%
% Note that the amsmath package sets \interdisplaylinepenalty to 10000
% thus preventing page breaks from occurring within multiline equations. Use:
%\interdisplaylinepenalty=2500
% after loading amsmath to restore such page breaks as IEEEtran.cls normally
% does. amsmath.sty is already installed on most LaTeX systems. The latest
% version and documentation can be obtained at:
% http://www.ctan.org/pkg/amsmath





% *** SPECIALIZED LIST PACKAGES ***
%
%\usepackage{algorithmic}
% algorithmic.sty was written by Peter Williams and Rogerio Brito.
% This package provides an algorithmic environment fo describing algorithms.
% You can use the algorithmic environment in-text or within a figure
% environment to provide for a floating algorithm. Do NOT use the algorithm
% floating environment provided by algorithm.sty (by the same authors) or
% algorithm2e.sty (by Christophe Fiorio) as the IEEE does not use dedicated
% algorithm float types and packages that provide these will not provide
% correct IEEE style captions. The latest version and documentation of
% algorithmic.sty can be obtained at:
% http://www.ctan.org/pkg/algorithms
% Also of interest may be the (relatively newer and more customizable)
% algorithmicx.sty package by Szasz Janos:
% http://www.ctan.org/pkg/algorithmicx




% *** ALIGNMENT PACKAGES ***
%
%\usepackage{array}
% Frank Mittelbach's and David Carlisle's array.sty patches and improves
% the standard LaTeX2e array and tabular environments to provide better
% appearance and additional user controls. As the default LaTeX2e table
% generation code is lacking to the point of almost being broken with
% respect to the quality of the end results, all users are strongly
% advised to use an enhanced (at the very least that provided by array.sty)
% set of table tools. array.sty is already installed on most systems. The
% latest version and documentation can be obtained at:
% http://www.ctan.org/pkg/array


% IEEEtran contains the IEEEeqnarray family of commands that can be used to
% generate multiline equations as well as matrices, tables, etc., of high
% quality.




% *** SUBFIGURE PACKAGES ***
%\ifCLASSOPTIONcompsoc
%  \usepackage[caption=false,font=normalsize,labelfont=sf,textfont=sf]{subfig}
%\else
%  \usepackage[caption=false,font=footnotesize]{subfig}
%\fi
% subfig.sty, written by Steven Douglas Cochran, is the modern replacement
% for subfigure.sty, the latter of which is no longer maintained and is
% incompatible with some LaTeX packages including fixltx2e. However,
% subfig.sty requires and automatically loads Axel Sommerfeldt's caption.sty
% which will override IEEEtran.cls' handling of captions and this will result
% in non-IEEE style figure/table captions. To prevent this problem, be sure
% and invoke subfig.sty's "caption=false" package option (available since
% subfig.sty version 1.3, 2005/06/28) as this is will preserve IEEEtran.cls
% handling of captions.
% Note that the Computer Society format requires a larger sans serif font
% than the serif footnote size font used in traditional IEEE formatting
% and thus the need to invoke different subfig.sty package options depending
% on whether compsoc mode has been enabled.
%
% The latest version and documentation of subfig.sty can be obtained at:
% http://www.ctan.org/pkg/subfig




% *** FLOAT PACKAGES ***
%
%\usepackage{fixltx2e}
% fixltx2e, the successor to the earlier fix2col.sty, was written by
% Frank Mittelbach and David Carlisle. This package corrects a few problems
% in the LaTeX2e kernel, the most notable of which is that in current
% LaTeX2e releases, the ordering of single and double column floats is not
% guaranteed to be preserved. Thus, an unpatched LaTeX2e can allow a
% single column figure to be placed prior to an earlier double column
% figure.
% Be aware that LaTeX2e kernels dated 2015 and later have fixltx2e.sty's
% corrections already built into the system in which case a warning will
% be issued if an attempt is made to load fixltx2e.sty as it is no longer
% needed.
% The latest version and documentation can be found at:
% http://www.ctan.org/pkg/fixltx2e


%\usepackage{stfloats}
% stfloats.sty was written by Sigitas Tolusis. This package gives LaTeX2e
% the ability to do double column floats at the bottom of the page as well
% as the top. (e.g., "\begin{figure*}[!b]" is not normally possible in
% LaTeX2e). It also provides a command:
%\fnbelowfloat
% to enable the placement of footnotes below bottom floats (the standard
% LaTeX2e kernel puts them above bottom floats). This is an invasive package
% which rewrites many portions of the LaTeX2e float routines. It may not work
% with other packages that modify the LaTeX2e float routines. The latest
% version and documentation can be obtained at:
% http://www.ctan.org/pkg/stfloats
% Do not use the stfloats baselinefloat ability as the IEEE does not allow
% \baselineskip to stretch. Authors submitting work to the IEEE should note
% that the IEEE rarely uses double column equations and that authors should try
% to avoid such use. Do not be tempted to use the cuted.sty or midfloat.sty
% packages (also by Sigitas Tolusis) as the IEEE does not format its papers in
% such ways.
% Do not attempt to use stfloats with fixltx2e as they are incompatible.
% Instead, use Morten Hogholm'a dblfloatfix which combines the features
% of both fixltx2e and stfloats:
%
% \usepackage{dblfloatfix}
% The latest version can be found at:
% http://www.ctan.org/pkg/dblfloatfix




% *** PDF, URL AND HYPERLINK PACKAGES ***
%
%\usepackage{url}
% url.sty was written by Donald Arseneau. It provides better support for
% handling and breaking URLs. url.sty is already installed on most LaTeX
% systems. The latest version and documentation can be obtained at:
% http://www.ctan.org/pkg/url
% Basically, \url{my_url_here}.




% *** Do not adjust lengths that control margins, column widths, etc. ***
% *** Do not use packages that alter fonts (such as pslatex).         ***
% There should be no need to do such things with IEEEtran.cls V1.6 and later.
% (Unless specifically asked to do so by the journal or conference you plan
% to submit to, of course. )



\begin{document}
%
% paper title
% Titles are generally capitalized except for words such as a, an, and, as,
% at, but, by, for, in, nor, of, on, or, the, to and up, which are usually
% not capitalized unless they are the first or last word of the title.
% Linebreaks \\ can be used within to get better formatting as desired.
% Do not put math or special symbols in the title.
\title{Recipe Pos Tagger\\ }


% author names and affiliations
% use a multiple column layout for up to three different
% affiliations
\author{\IEEEauthorblockN{Mehmet �zgen\IEEEauthorrefmark{1},
{\.I}brahim Ard{\i}�\IEEEauthorrefmark{2}, G�nen� Ercan\IEEEauthorrefmark{3} and
P{\i}nar Duygulu {\c{S}}ahin\IEEEauthorrefmark{4}}
\IEEEauthorblockA{Department of Computer Engineering,
Hacettepe University\\
Ankara\\
Email: \IEEEauthorrefmark{1}mehmet.ozgen@hacettepe.edu.tr,
\IEEEauthorrefmark{2}ibrahim.ardic@hacettepe.edu.tr,
\IEEEauthorrefmark{3}gonenc.ercan@hacettepe.edu.tr,
\IEEEauthorrefmark{4}pduygulu@hacettepe.edu.tr}}

% conference papers do not typically use \thanks and this command
% is locked out in conference mode. If really needed, such as for
% the acknowledgment of grants, issue a \IEEEoverridecommandlockouts
% after \documentclass

% for over three affiliations, or if they all won't fit within the width
% of the page, use this alternative format:
% 
%\author{\IEEEauthorblockN{Michael Shell\IEEEauthorrefmark{1},
%Homer Simpson\IEEEauthorrefmark{2},
%James Kirk\IEEEauthorrefmark{3}, 
%Montgomery Scott\IEEEauthorrefmark{3} and
%Eldon Tyrell\IEEEauthorrefmark{4}}
%\IEEEauthorblockA{\IEEEauthorrefmark{1}School of Electrical and Computer Engineering\\
%Georgia Institute of Technology,
%Atlanta, Georgia 30332--0250\\ Email: see http://www.michaelshell.org/contact.html}
%\IEEEauthorblockA{\IEEEauthorrefmark{2}Twentieth Century Fox, Springfield, USA\\
%Email: homer@thesimpsons.com}
%\IEEEauthorblockA{\IEEEauthorrefmark{3}Starfleet Academy, San Francisco, California 96678-2391\\
%Telephone: (800) 555--1212, Fax: (888) 555--1212}
%\IEEEauthorblockA{\IEEEauthorrefmark{4}Tyrell Inc., 123 Replicant Street, Los Angeles, California 90210--4321}}




% use for special paper notices
%\IEEEspecialpapernotice{(Invited Paper)}




% make the title area
\maketitle

% As a general rule, do not put math, special symbols or citations
% in the abstract
\begin{abstract}
The instruction list is a key for a success of doing
any task correctly. The validation of correction of the task by
examined only the result of the performed instructions, actions.
But on the other hand, when performing the actions according to
the instructions under the supervision of domain expert, there is
high probability to detect disruptions and errors without waiting
the result of the task. In this context, mapping instructions to
actions is getting high importance when the domain is one of
safety-critical systems, physiotherapy exercises, food recipes,
chemical experiments, etc. In this work, any written instructions
in a specific domain are learned using natural language
processing techniques (word2vec, CRF++) and will generate a
model to use this information for revealing hidden objects in
intermediate states during cooking. 
\end{abstract}

% no keywords

\begin{IEEEkeywords} Information Retrieval, Natural Language
Processing, weaky-labelled data, text analysis, big data \end{IEEEkeywords}


% For peer review papers, you can put extra information on the cover
% page as needed:
% \ifCLASSOPTIONpeerreview
% \begin{center} \bfseries EDICS Category: 3-BBND \end{center}
% \fi
%
% For peerreview papers, this IEEEtran command inserts a page break and
% creates the second title. It will be ignored for other modes.
\IEEEpeerreviewmaketitle



\section{Introduction}
% no \IEEEPARstart
% XeLaTeX can use any Mac OS X font. See the setromanfont command below.
% Input to XeLaTeX is full Unicode, so Unicode characters can be typed directly into the source.

% The next lines tell TeXShop to typeset with xelatex, and to open and save the source with Unicode encoding.

%!TEX TS-program = xelatex
%!TEX encoding = UTF-8 Unicode

\documentclass[12pt]{article}
\usepackage{geometry}                % See geometry.pdf to learn the layout options. There are lots.
\geometry{letterpaper}                   % ... or a4paper or a5paper or ... 
%\geometry{landscape}                % Activate for for rotated page geometry
%\usepackage[parfill]{parskip}    % Activate to begin paragraphs with an empty line rather than an indent
\usepackage{graphicx}
\usepackage{amssymb}

% Will Robertson's fontspec.sty can be used to simplify font choices.
% To experiment, open /Applications/Font Book to examine the fonts provided on Mac OS X,
% and change "Hoefler Text" to any of these choices.

\usepackage{fontspec,xltxtra,xunicode}
\defaultfontfeatures{Mapping=tex-text}
\setromanfont[Mapping=tex-text]{Hoefler Text}
\setsansfont[Scale=MatchLowercase,Mapping=tex-text]{Gill Sans}
\setmonofont[Scale=MatchLowercase]{Andale Mono}

\title{Brief Article}
\author{The Author}
%\date{}                                           % Activate to display a given date or no date

\begin{document}
\maketitle

% For many users, the previous commands will be enough.
% If you want to directly input Unicode, add an Input Menu or Keyboard to the menu bar 
% using the International Panel in System Preferences.
% Unicode must be typeset using a font containing the appropriate characters.
% Remove the comment signs below for examples.

% \newfontfamily{\A}{Geeza Pro}
% \newfontfamily{\H}[Scale=0.9]{Lucida Grande}
% \newfontfamily{\J}[Scale=0.85]{Osaka}

% Here are some multilingual Unicode fonts: this is Arabic text: {\A ?????? ?????}, this is Hebrew: {\H ????}, 
% and here's some Japanese: {\J ???}.



\end{document}  \documentclass[11pt, oneside]{article}   	% use "amsart" instead of "article" for AMSLaTeX format
\usepackage{geometry}                		% See geometry.pdf to learn the layout options. There are lots.
\geometry{letterpaper}                   		% ... or a4paper or a5paper or ... 
%\geometry{landscape}                		% Activate for rotated page geometry
%\usepackage[parfill]{parskip}    		% Activate to begin paragraphs with an empty line rather than an indent
\usepackage{graphicx}				% Use pdf, png, jpg, or eps§ with pdflatex; use eps in DVI mode
								% TeX will automatically convert eps --> pdf in pdflatex		
\usepackage{amssymb}

%SetFonts

%SetFonts


\title{Brief Article}
\author{The Author}
%\date{}							% Activate to display a given date or no date

\begin{document}
\maketitle
%\section{}
%\subsection{}



\end{document}  \cite{coreNlp}



\subsection{Subsection Heading Here}
Subsection text here.


\subsubsection{Subsubsection Heading Here}
Subsubsection text here.


\section{Related Works}
Algorithmic processing of training texts and / or commands expressed in natural language has long been an interesting question (Harnad, 1990). In this problem, the artificial intelligence agent (such as a robot) is trying to learn the operations corresponding to symbols that are expressed in a field. The system is expected to automatically interact with definitions in the physical world as symbols to interact with objects and objects. We obtained through www.allrecipes.com

Chen and Mooney (2011) have developed a system that can act according to the expressions expressed in natural language in the system they developed. Based on the reinforced learning methods, the system can translate the sentences into natural language by using the information in the location (such as the objects on the wall) and the state flow that is modeled. As can be seen from these studies in the literature, reinforcement learning techniques are used in problems where the current situation can be modeled more clearly.
A slightly different problem is Kushman et al. (2014) developed a similar method for solving problems expressed in mathematical texts. According to this method, the system is mapping the problem described in the text to a set of equations and producing the result. To learn the method, we use a training set marked at different levels. Equations that can be created in the system can be given or can be learned only when the last answer is given. In the method they developed, the solution is defined as diagrams, and a probability model that maps the information given in the text to the diagram is developed. Of course, it is necessary that the structure of the questionnaire is similar and that all of them can be transferred to the equation diagram.

Recipes can also be thought of as a set of steps that must be followed. From this point of view, there are common aspects to the problems described above. The most important difference is that it is difficult to convert the media knowledge to a reward function to provide reinforcement learning. The biggest reason for this is that it is difficult to model the correct sequence of operations that can be done while cooking. To overcome this problem in the processing of recipes, three different basic approaches are striking. The first is knowledge-based methods that are dependent on previously created knowledge vocabulary, the second is those using supervised learning methods, and the third is unsupervised learning and extracting a specific model from the data in the compilation. The first approach involves all the recipes, contents, tools and processes, and methods that can show the relationship between them. In the second approach, it is necessary to generate the data with the equivalent of the text for the training compilation. In the third category, transductive learning is performed according to the solutions given, ie it is not possible to classify the given samples and to process new ones.

\subsection{Information Based Methods}
Walter et al. (2011) describes a method of generating flow diagrams from recipe texts, extracting recipe statements as material, processing and finishing conditions. Based on this information, a flow diagram is created. This method requires pre-tagged sentences and a dictionary that holds all of these tags.

Another problem that may be related is the creation of an information dictionary. Gaiilard et al. (2012) foods and their relationships with a Wiki-style method. Its approach basically consists of 6 hierarchies; diet and mealtime according to different features categorize the food. With this approach it is possible to adapt a new recipe or an existing recipe.

Although food-related ontology is increasingly detailed, these methods often fail to adapt to dynamic language and variety. Thus, when ontology methods are used alone, successful results are achieved for some data sets, but success is often low. Moreover, in order to adapt these methods to a different language, ontologies should be structured in accordance with other languages.

\subsection{Supervised Learning Methods}
One of the most important features of tutorial learning techniques is the target description model. Basically, the information that this model needs to be able to show is where the processes described in the steps are done on which materials and where the output of this step is used. Some of these associations show in the tree structure (Jermsurawong and Habash, 2015), while some researchers show it in a non-cyclic manner (Malmoud et al., 2014). It is possible that this notation can both express all the recipes and learn it automatically.
Jermsurawong and Habash (2015) defined a representation in the tree data structure that stores food items and their relationships over the description steps. This demonstration can show which step of the recipes is related to which steps and materials.

Malmoud et al. (2014) sees text in recipes as a Markov Decision Process problem by expanding semantic role labeling. It holds two things between the process and the material. The purpose of this relationship is to demonstrate the processes within the training that cause these situations to occur.
Mori et al. (2014) created a labeling tool by showing recipes as non-cyclic charts in their initial work. With this tool, the recipe created a graph that indicates the skis over the text. The given Japanese recipes first extract the word segmentation, locate the words in the clan's tasks, label the named entities, and finally construct the predicate-argument structure.
In order to resolve the uncertainties in the contents, Erica Greene (2016) tagged a total of 187,000 sentences for labeling and edited the tags of the words in the table of contents using the Conditional Random Fields method.


\subsection{Unsupervised Learning Methods}
In 2015, Kiddon et al. (2015) has developed a method of learning without a teacher for the processing of recipes. Unlike previous works, this method learns the linkages of the line by using different parameters according to the variables in the system and using the expectation maximization method. Since this model is constructed according to the definition in the drawing, it is not possible to learn the features of an extragalactic recipe. In this respect, it is a transduction method, all the recipes to be processed have to be obtained at the model learning time. Moreover, the generated data model has a high number of implicit variables because it defines both the verbs, the contents, and the content of the contents in relation to the steps as one of the parameters of the model. When such high numbered models are learned, they can be found in a high number of local maxima, which prevents the optimum model from being found.
It is intended to remove them as to which action is associated with the material, removing the flow of the recipe as a whole. These methods are inspired by semantic spaces used in text mining and meaning spaces that show word meaning relations. By using these material spaces, it is possible to determine which materials can be replaced instead of which materials. Nedovic (2013) used a similar method to define a method of learning materials for different types of food. Latent Dirichlet Allocation (LDA) and Deep Belief Networks (DBN) are used in the method. It has been seen that the output of the system can group materials according to the foods in different kitchens. A similar study has been proposed by Achananuparp and Weber (2016) to produce safer food recommendations in meals. In this method, Singular Value Decomposition (SVD) method, which is widely used for extracting word meaning relations, is used.

\section{Experimental Setup}
As seen in the literature, there are many methods aiming at revealing the relationship between the description texts and the contents of the description texts, the sequence of the process flow and the situations that occur during the actions and showing them with the models by many different methods. The texts should be easily perceivable by the computer by passing 1830 recipes we obtained through www.alrecipes.com through certain operations.

Using NLTK libraries, each of the recipes was first converted into a clean text by separating individual recipes and separating them into words, punctuation, and meaningless words.  (names, categories, contents, descriptions, and comments). Then each recipe is divided into sentences for labelling.

\subsection{Labelling}
Using the NLTK library, labeling was done according to the culled state (VB, NN, ADJ, etc.) that each of the belts passed. As a result of the labeling made, "I can chopped green chile peppers" in the table of contents is labeled as below in TABLE 1.

\begin{table}[]
\centering
\caption{Tags}
\label{my-label}
\begin{tabular}{|l|l|l|l|l|l|}
\hline
I   & can & chopped & green & chile & peppers \\ \hline
ADJ & NN  & VB      & NN    & NN    & NN      \\ \hline
\end{tabular}
\end{table}

As can be seen in Table 1, it does not allow me to know that the word "life" is a unit of measure, "1" actually tells the quantity, and that the words "green" and "choped" are actually interpretations of a "papers" word.

The challenge of parsing the recipe is to be able to distinguish content components from component cues. Erica Greene (2016) has trained 171,244 sets (labeled as UNIT, QUANTITY, COMMENT and OTHER) with the very specific set of CRF ++ (Conditional Random Fields) method she has created to solve this problem and has been able to label the contents portion probabilistically with a newly given description. Let us have the sentence "1 teaspoon sugar". The model is using 171,224 tagged data to learn a model that can predict the tag sequence for any sentence we have given to it, even though we have never seen this component count before. It approaches this by modeling the conditional probability of a set of labels.

p (UNIT UNIT UNIT| "1 teaspoon sugar")

p (QUANTITY UNIT UNIT| "1 teaspoon sugar")

p (UNIT QUANTITY UNIT| "1 teaspoon sugar")

p (UNIT UNIT QUANTITY "1 teaspoon sugar")

p (UNIT QUANTITY QUANTITY "1 teaspoon sugar")

p (QUANTITY QUANTITY QUANTITY | "1 teaspoon sugar")

p (UNIT QUANTITY NAME| "1 teaspoon sugar")

...

As mentioned above, it calculates all the probabilities that can be labeled "1 teaspoon sugar". The beauty of the linear-chain CRF model makes some conditional independence assumptions that allow us to use dynamic programming to efficiently search the area of all possible label sequences. As a result, we have re-tagged our data with Erica Greene (2016), and the result is shown in TABLE 2. The results are shown in Table 2, which shows the best label sequence at a time that is linear with the number of second.


\begin{table}[]
\centering
\caption{Tags}
\label{my-label}
\begin{tabular}{|l|l|l|l|l|l|}
\hline
I   & can & chopped & green & chile & peppers \\ \hline
QT & UNIT  & CMMT      & CMMT    & NAME    & NAME      \\ \hline
\end{tabular}
\end{table}

It was observed that 450 of 171.224 labeled samples were separated and tested and 76 percentage correctly tagged.


\subsection{Calculation of Proximity}
Kiddon et al. (2015), some words are not in the table of contents, but more than one is meant. (Eg mixture, them) Kiddon et al. (2015) defines this as a hidden object. He has created a probabilistic model to make hidden objects clear. In order to reveal these hidden objects, we present a relationship between the words 'NAME' in this action and its contents. In fact, each word is represented as a vector and we calculate the cosine similarity between them.

Using the Word2Vec library that displays 3.5 billion words created by a working group led by Tomas Mikolov (Google) as a 300-dimensional vector vocabulary, the words that pass through the intellectual part and are labeled 'NAME' have been converted into a 300-dimensional vector.

In the recipe sentence set, phrases without any word labeled 'NAME' are removed. When the words labeled as 'VB' are extracted, all the words are converted into 300 dimensional vectors and the averages are taken. Because the other words around us can also provide us with information about what materials the hidden object contains. If you only go through the actions, you will be made a comment by looking at the similarity of only two calves. However, the sentence can sometimes contain words that characterize the words in the contents. Let's take a look at "Cook the mixture until caramelized". It is labeled as "cook-VB". When we look at the similarity of the cosine with the contents, it is seen that almost all of them resemble to each other. However, when the complex average of c�mlena is taken, "caramelized" word is included. And the result is more similar to "Onion" and "sugar". Taking the average of the blame for this reason actually brings us closer to the right conclusion.

% An example of a floating figure using the graphicx package.
% Note that \label must occur AFTER (or within) \caption.
% For figures, \caption should occur after the \includegraphics.
% Note that IEEEtran v1.7 and later has special internal code that
% is designed to preserve the operation of \label within \caption
% even when the captionsoff option is in effect. However, because
% of issues like this, it may be the safest practice to put all your
% \label just after \caption rather than within \caption{}.
%
% Reminder: the "draftcls" or "draftclsnofoot", not "draft", class
% option should be used if it is desired that the figures are to be
% displayed while in draft mode.
%
%\begin{figure}[!t]
%\centering
%\includegraphics[width=2.5in]{myfigure}
% where an .eps filename suffix will be assumed under latex, 
% and a .pdf suffix will be assumed for pdflatex; or what has been declared
% via \DeclareGraphicsExtensions.
%\caption{Simulation results for the network.}
%\label{fig_sim}
%\end{figure}

% Note that the IEEE typically puts floats only at the top, even when this
% results in a large percentage of a column being occupied by floats.


% An example of a double column floating figure using two subfigures.
% (The subfig.sty package must be loaded for this to work.)
% The subfigure \label commands are set within each subfloat command,
% and the \label for the overall figure must come after \caption.
% \hfil is used as a separator to get equal spacing.
% Watch out that the combined width of all the subfigures on a 
% line do not exceed the text width or a line break will occur.
%
%\begin{figure*}[!t]
%\centering
%\subfloat[Case I]{\includegraphics[width=2.5in]{box}%
%\label{fig_first_case}}
%\hfil
%\subfloat[Case II]{\includegraphics[width=2.5in]{box}%
%\label{fig_second_case}}
%\caption{Simulation results for the network.}
%\label{fig_sim}
%\end{figure*}
%
% Note that often IEEE papers with subfigures do not employ subfigure
% captions (using the optional argument to \subfloat[]), but instead will
% reference/describe all of them (a), (b), etc., within the main caption.
% Be aware that for subfig.sty to generate the (a), (b), etc., subfigure
% labels, the optional argument to \subfloat must be present. If a
% subcaption is not desired, just leave its contents blank,
% e.g., \subfloat[].


% An example of a floating table. Note that, for IEEE style tables, the
% \caption command should come BEFORE the table and, given that table
% captions serve much like titles, are usually capitalized except for words
% such as a, an, and, as, at, but, by, for, in, nor, of, on, or, the, to
% and up, which are usually not capitalized unless they are the first or
% last word of the caption. Table text will default to \footnotesize as
% the IEEE normally uses this smaller font for tables.
% The \label must come after \caption as always.
%
%\begin{table}[!t]
%% increase table row spacing, adjust to taste
%\renewcommand{\arraystretch}{1.3}
% if using array.sty, it might be a good idea to tweak the value of
% \extrarowheight as needed to properly center the text within the cells
%\caption{An Example of a Table}
%\label{table_example}
%\centering
%% Some packages, such as MDW tools, offer better commands for making tables
%% than the plain LaTeX2e tabular which is used here.
%\begin{tabular}{|c||c|}
%\hline
%One & Two\\
%\hline
%Three & Four\\
%\hline
%\end{tabular}
%\end{table}


% Note that the IEEE does not put floats in the very first column
% - or typically anywhere on the first page for that matter. Also,
% in-text middle ("here") positioning is typically not used, but it
% is allowed and encouraged for Computer Society conferences (but
% not Computer Society journals). Most IEEE journals/conferences use
% top floats exclusively. 
% Note that, LaTeX2e, unlike IEEE journals/conferences, places
% footnotes above bottom floats. This can be corrected via the
% \fnbelowfloat command of the stfloats package.




\section{Conclusion}
The conclusion goes here.




% conference papers do not normally have an appendix


% use section* for acknowledgment
\section*{Acknowledgment}


The authors would like to thank...





% trigger a \newpage just before the given reference
% number - used to balance the columns on the last page
% adjust value as needed - may need to be readjusted if
% the document is modified later
%\IEEEtriggeratref{8}
% The "triggered" command can be changed if desired:
%\IEEEtriggercmd{\enlargethispage{-5in}}

% references section

% can use a bibliography generated by BibTeX as a .bbl file
% BibTeX documentation can be easily obtained at:
% http://mirror.ctan.org/biblio/bibtex/contrib/doc/
% The IEEEtran BibTeX style support page is at:
% http://www.michaelshell.org/tex/ieeetran/bibtex/
%\bibliographystyle{IEEEtran}
% argument is your BibTeX string definitions and bibliography database(s)
%\bibliography{IEEEabrv,../bib/paper}
%
% <OR> manually copy in the resultant .bbl file
% set second argument of \begin to the number of references
% (used to reserve space for the reference number labels box)

\bibliographystyle{plain}
\bibliography{myRef}


% that's all folks
\end{document}


